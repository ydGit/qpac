\graphicspath{{../05QuantumPhysics/pics/}}

\chapter{Quantum Physics}\label{ch:QuantumPhysics}
\lettrine[lines=2]{\color{darkocre}T}{he} first type of operators -- and
corresponding tensors -- that we encountered has a simple type:
\[
\op{L}\,\vec{a} = \vec{b}\,.
\]
It is a linear unary function mapping vectors into vectors.


\begin{myprereq}{Prerequisite Knowledge}
To fully understand the material of this chapter, readers should be comfortable with the following concepts:

\begin{itemize}
	\item \phantom{phantom}
	\vspace{-0.5cm}
	\item State
	\item Dynamical equations
\end{itemize}	
\end{myprereq}

\section{Quantum System}\label{sec:QuantumSystem}
We are looking for a binary operator $\op{\sigma}$ that yields a number
based on two vectors:
\[
\ketbra{\sigma}{\sigma}\,\vec{a}\,\vec{b}=x\,.
\]

\section{Quantum State}\label{sec:QuantumState}
We are looking for a binary operator $\op{\sigma}$ that yields a number
based on two vectors:
\[
\ketbra{\sigma}{\sigma}\,\vec{a}\,\vec{b}=x\,.
\]
\subsection{States Overlap}
\[
\braket{\psi}{\phi}.
\]

\section{Quantum Dynamics}\label{sec:QuantumDynamics}
We are looking for a binary operator $\op{\sigma}$ that yields a number
based on two vectors:
\[
\ketbra{\sigma}{\sigma}\,\vec{a}\,\vec{b}=x\,.
\]

\section{Quantum Hamiltonian}\label{sec:QuantumHamiltonian}
We are looking for a binary operator $\op{\sigma}$ that yields a number
based on two vectors:
\[
\ketbra{\sigma}{\sigma}\,\vec{a}\,\vec{b}=x\,.
\]

\section{Quantum Oscillator}\label{sec:QuantumOscillator}
We are looking for a binary operator $\op{\sigma}$ that yields a number
based on two vectors:
\[
\ketbra{\sigma}{\sigma}\,\vec{a}\,\vec{b}=x\,.
\]

\section{Quantum Bit}\label{sec:Qubit}
We are looking for a binary operator $\op{\sigma}$ that yields a number
based on two vectors:
\[
\ketbra{\sigma}\,\vec{a}\,\vec{b}=x\,.
\]
\subsection{Physical Realization of Qubits}
Recall that harmonic oscillator is any physical system with Hamiltonian
\[
H = \frac{p^2}{2m}+\frac{kx^2}{2}\,.
\] 
Many concrete physical systems can be described using this Hamiltonian and thus provide specific \emph{realizations} of 
the oscillator model. Similarly, many concrete physical systems realize the idea of a qubit.

\section{Interacting Qubits}\label{sec:InteractingQubits}
We are looking for a binary operator $\op{\sigma}$ that yields a number
based on two vectors:
\[
\ketbra{\sigma}{\sigma}\,\vec{a}\,\vec{b}=x\,.
\]
\subsection{Computational Basis}
\[
\ket{\Upsilon}_1 = \ket{0}\ket{0},\,\ket{\Upsilon}_2 = \ket{0}\ket{1},\,
\ket{\Upsilon}_3 = \ket{1}\ket{0},\,\ket{\Upsilon}_4 = \ket{1}\ket{1}\,.
\]
Q: Are there other states, which are also basis and product? Smth like
\[
\ket{\Xi}=\ket{+}\ket{+}\,.
\]

\subsection{Bell States}
\[
\ket{\Phi}^{+}\,,\quad\ket{\Phi}^{-}\,,\quad
\ket{\Psi}^{+}\,,\quad\ket{\Psi}^{-}\,.
\]


\subsection{GHZ State}\label{sec:GHZState}
We are looking for a binary operator $\op{\sigma}$ that yields a number
based on two vectors:
\[
\ketbra{\sigma}{\sigma}\,\vec{a}\,\vec{b}=x\,.
\]


\section{Quantum Field}\label{sec:QuantumField}
We are looking for a binary operator $\op{\sigma}$ that yields a number
based on two vectors:
\[
\ketbra{\sigma}{\sigma}\,\vec{a}\,\vec{b}=x\,.
\]

We are looking for a binary operator $\op{\sigma}$ that yields a number
based on two vectors:
\[
\op{\sigma}\,\vec{a}\,\vec{b}=x\,.
\]
We will call this operator $\op{\sigma}$ \emph{dol}-operator\footnote{This is not a
standard terminology. }, based on the key letters of the phrase
``\underline{d}egree of \underline{o}ver\underline{l}ap''.

\begin{myrem}{Reminder}
When we say that an operator $\op{\Gamma}$ is given or known, we
mean that we know how it acts on \emph{any vector} $\vec{a}$:
\[
\op{\Gamma}\,\vec{a} = x_a\,.
\]
\end{myrem}

Array of equations:
\begin{eqnarray}
  \op{\Gamma}_1\,\vec{e}_1 & = & 1\,\\
  \op{\Gamma}_1\,\vec{e}_2 & = & 0\,\\
  \op{\Gamma}_1\,\vec{e}_3 & = & 0\,\\
  \ldots
\end{eqnarray}

\section*{Chapter Highlights}
{\setstretch{1.5}\chhc
  \it
\begin{itemize}
\item Two vectors can be compared for similarity by calculating the
  ``degree of overlap''. The longer two vectors are and the closer
  their mutual direction -- the greater the overlap is.
\item Degree of overlap can be described by a binary linear operator
  $\op{\sigma}$. This operator is closely related to the concept of
  scalar product of two vectors.
\item When scalar product (or, equivalently, degree of overlap) is
  defined for vectors, each vector receives a ``special relative'' --
  conjugate vector -- that lives in different vector space, called
  conjugate or dual space.
\item When the degree-of-overlap operator $\op{\sigma}$ is partially
  applied, the result is a unary linear operator that yields a number
  for every input vector. Importantly, such an operator is also a
  vector, albeit not an arrow-like vector.
\end{itemize}

}
