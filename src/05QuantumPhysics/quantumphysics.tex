\graphicspath{{../05QuantumPhysics/pics/}}

\chapter{Quantum Physics}\label{ch:QuantumPhysics}
\lettrine[lines=2]{\color{darkocre}T}{he} first type of operators -- and
corresponding tensors -- that we encountered has a simple type:
\[
\op{L}\,\vec{a} = \vec{b}\,.
\]
It is a linear unary function mapping vectors into vectors.


\begin{myprereq}{Prerequisite Knowledge}
To fully understand the material of this chapter, readers should be comfortable with the following concepts:

\begin{itemize}
	\item \phantom{phantom}
	\vspace{-0.5cm}
	\item State
	\item Dynamical equations
\end{itemize}	
\end{myprereq}

\section{Quantum System}\label{sec:QuantumSystem}
We are looking for a binary operator $\op{\sigma}$ that yields a number
based on two vectors:
\[
\ketbra{\sigma}{\sigma}\,\vec{a}\,\vec{b}=x\,.
\]

\section{Quantum State}\label{sec:QuantumState}
We are looking for a binary operator $\op{\sigma}$ that yields a number
based on two vectors:
\[
\ketbra{\sigma}{\sigma}\,\vec{a}\,\vec{b}=x\,.
\]
\subsection{States Overlap}
\[
\braket{\psi}{\phi}.
\]

\section{Quantum Dynamics}\label{sec:QuantumDynamics}
We are looking for a binary operator $\op{\sigma}$ that yields a number
based on two vectors:
\[
\ketbra{\sigma}{\sigma}\,\vec{a}\,\vec{b}=x\,.
\]

\section{Quantum Hamiltonian}\label{sec:QuantumHamiltonian}
We are looking for a binary operator $\op{\sigma}$ that yields a number
based on two vectors:
\[
\ketbra{\sigma}{\sigma}\,\vec{a}\,\vec{b}=x\,.
\]


\section{Quantum Bit}\label{sec:Qubit}
Any quantum system with two active states is called a \emph{qubit}. The state with lower energy is usually called \emph{ground state} and denoted as $\ket{g}$ or $\ket{0}$ (zero). The state with higher energy is usually called \emph{excited state} and denoted as $\ket{e}$ or $\ket{1}$ (one). The notation $\ket{0}\,,\ket{1}$ is used in the field of quantum information and computation.

If the energy of the ground and excited states are $E_g$ and $E_e$, respectively, then the Hamiltonian of a qubit can be written using projectors
\[
\op{H} = E_g\ketbra{g}{g}+E_e\ketbra{e}{e}\,.
\]
It requires an energy $\Delta E=E_e-E_g$ to excite the qubit from the lower energy state to the the higher energy state. This energy may come from a quantum of electromagnetic field oscillating with frequency $\omega=\Delta E/\hbar$.

\subsection{Flipping Operators}
Transition between the states of a qubit can be described mathematically using operators that map one state into another. For example, an operator $\op{F}$ that \emph{flips} states acts must do the following:
\[
\op{F}\,\ket{0}=\ket{1}\,,
\]
\[
\op{F}\,\ket{1}=\ket{0}\,.
\]
Such an operator can be easily built from the tensor products:
\[
\op{F} = \ketbra{1}{0}+\ketbra{0}{1}\,.
\]
Each term in this sum is useful in quantum theory. The first term is called \emph{raising operator} and is denoted as 
$ \op{\sigma}_\plus=\ketbra{1}{0}$. The second term is called \emph{lowering operator} and is denoted as $ \op{\sigma}_\minus=\ketbra{0}{1}$. Apparently, the raising operator excites the qubit from the ground state, while the lowering operator brings the qubit down from the excited state.

\begin{exercise}
	Calculate (a) $\op{\sigma}_\plus \op{\sigma}_\plus$; (b) $\op{\sigma}_\minus \op{\sigma}_\minus$; (c) $\op{\sigma}_\plus \op{\sigma}_\minus$; (d) $\op{\sigma}_\minus \op{\sigma}_\plus$.
\end{exercise}

\begin{exercise}
	Show that  $\op{\sigma}_\plus \op{\sigma}_\minus+\op{\sigma}_\minus \op{\sigma}_\plus=\op{I}$, where $\op{I}$ is the identity operator.
\end{exercise}

\begin{exercise}
	Show that the qubit Hamiltonian can be written in terms of the raising and lowering operators as follows:
	
	\[
	\op{H} = \hbar\omega\left(\op{\sigma}_\plus \op{\sigma}_\minus+\epsilon\op{I} \right)\,,
	\]
	where $\epsilon=E_g/\Delta E$.
\end{exercise}

\subsection{Number Operator}
The operator $\op{n}=\op{\sigma}_\plus \op{\sigma}_\minus$ is called \emph{number operator} for the following reason. 

\section{Quantum Oscillator}\label{sec:QuantumOscillator}
We are looking for a binary operator $\op{\sigma}$ that yields a number
based on two vectors:
\[
\ketbra{\sigma}{\sigma}\,\vec{a}\,\vec{b}=x\,.
\]



\subsection{Physical Realization of Qubits}
Recall that harmonic oscillator is any physical system with Hamiltonian
\[
H = \frac{p^2}{2m}+\frac{kx^2}{2}\,.
\] 
Many concrete physical systems can be described using this Hamiltonian and thus provide specific \emph{realizations} of 
the oscillator model. Similarly, many concrete physical systems realize the idea of a qubit.

\section{Interacting Qubits}\label{sec:InteractingQubits}
We are looking for a binary operator $\op{\sigma}$ that yields a number
based on two vectors:
\[
\ketbra{\sigma}{\sigma}\,\vec{a}\,\vec{b}=x\,.
\]
\subsection{Computational Basis}
\[
\ket{\Upsilon}_1 = \ket{0}\ket{0},\,\ket{\Upsilon}_2 = \ket{0}\ket{1},\,
\ket{\Upsilon}_3 = \ket{1}\ket{0},\,\ket{\Upsilon}_4 = \ket{1}\ket{1}\,.
\]
Q: Are there other states, which are also basis and product? Smth like
\[
\ket{\Xi}=\ket{+}\ket{+}\,.
\]

\subsection{Bell States}
\[
\ket{\Phi}^{+}\,,\quad\ket{\Phi}^{-}\,,\quad
\ket{\Psi}^{+}\,,\quad\ket{\Psi}^{-}\,.
\]


\subsection{GHZ State}\label{sec:GHZState}
We are looking for a binary operator $\op{\sigma}$ that yields a number
based on two vectors:
\[
\ketbra{\sigma}{\sigma}\,\vec{a}\,\vec{b}=x\,.
\]


\section{Quantum Field}\label{sec:QuantumField}
We are looking for a binary operator $\op{\sigma}$ that yields a number
based on two vectors:
\[
\ketbra{\sigma}{\sigma}\,\vec{a}\,\vec{b}=x\,.
\]

We are looking for a binary operator $\op{\sigma}$ that yields a number
based on two vectors:
\[
\op{\sigma}\,\vec{a}\,\vec{b}=x\,.
\]
We will call this operator $\op{\sigma}$ \emph{dol}-operator\footnote{This is not a
standard terminology. }, based on the key letters of the phrase
``\underline{d}egree of \underline{o}ver\underline{l}ap''.

\begin{myrem}{Reminder}
When we say that an operator $\op{\Gamma}$ is given or known, we
mean that we know how it acts on \emph{any vector} $\vec{a}$:
\[
\op{\Gamma}\,\vec{a} = x_a\,.
\]
\end{myrem}

Array of equations:
\begin{eqnarray}
  \op{\Gamma}_1\,\vec{e}_1 & = & 1\,\\
  \op{\Gamma}_1\,\vec{e}_2 & = & 0\,\\
  \op{\Gamma}_1\,\vec{e}_3 & = & 0\,\\
  \ldots
\end{eqnarray}

\section*{Chapter Highlights}
{\setstretch{1.5}\chhc
  \it
\begin{itemize}
\item Two vectors can be compared for similarity by calculating the
  ``degree of overlap''. The longer two vectors are and the closer
  their mutual direction -- the greater the overlap is.
\item Degree of overlap can be described by a binary linear operator
  $\op{\sigma}$. This operator is closely related to the concept of
  scalar product of two vectors.
\item When scalar product (or, equivalently, degree of overlap) is
  defined for vectors, each vector receives a ``special relative'' --
  conjugate vector -- that lives in different vector space, called
  conjugate or dual space.
\item When the degree-of-overlap operator $\op{\sigma}$ is partially
  applied, the result is a unary linear operator that yields a number
  for every input vector. Importantly, such an operator is also a
  vector, albeit not an arrow-like vector.
\end{itemize}

}
