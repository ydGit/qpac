This book is the result of lectures delivered to curious, motivated, and studious high schoolers. The lectures ran  during the years 2019-2024 in various formats, but mostly in class during a three week summer school organized by Columbia University Pre-College Programs. Additionally, the same lectures were taught remotely to selected students of Ukrainian Physics and Mathematics Lyceum. 

The material has been designed to be accessible to people with solid background in high-school algebra and physics (mostly mechanics). Several years of teaching to a relatively diverse set of students proved that nearly all material can be efficiently absorbed by most, provided diligent work is done on exercises and problem. The last fact confirms a well-known truism: \emph{No real learning occurs without practice.}

Exercises are essential part of this book.  They are carefully selected to help readers get better understanding of the material and they are also fully solved. The difficulty of the exercises varies from simple to quite challenging.

This book \emph{is not a standard textbook} and it lacks the breadth and rigor present in many excellent introductions into Quantum Physics. The best way to view this book is as a \emph{bridge} between  elementary and popular books and the more challenging college-level textbooks. 

\section*{At Any Cost}
To explain the subtitle of this book let us refer to the letter written by Max Karl Ernst Ludwig Planck to an American physicist

This book explores \emph{tensors} -- a type of mathematical objects
that extends the notion of numbers and vectors. The method of
exploration is deliberately chosen to resemble a journey. Starting
from familiar grounds of numbers and operations with numbers, a reader
will re-examine familiar concepts in a new light and then will arrive
at new concepts gradually, connecting the dots along the way.

Although the topic of the book is mathematical, the exploration will
lack proper mathematical rigor, aiming instead at simplicity, clarity,
and the use of helpful analogies.

This book is {\bf not intended} to substitute more serious textbooks on linear
algebra or tensor algebra. Hopefully, the main benefit of
reading this book -- either before, or after, or in addition to other
books on the subject -- is that it should help lower the
\emph{``mental barrier''}\index{Barrier!mental} we all encounter when
learning new concepts, especially abstract mathematical concepts.

To comprehend and enjoy the material of this book the reader should
have a solid knowledge of basic high-school algebra and an open and
inquiring mind. The book is a bit longer that it could have been
because all derivations are detailed and all exercises are fully
solved.

Some sections are marked with an asterisk, for example
{\bf Transposition*}. Those sections contain material that is either
optional or a bit more advanced that usual. These sections can be
skipped without significant impact on the main message of the book.
