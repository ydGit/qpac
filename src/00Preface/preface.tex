This book is the result of lectures delivered to curious, motivated, and studious high schoolers. The lectures ran  during the years 2019-2024 in various formats, but mostly in class during a three week summer school organized by Columbia University Pre-College Programs. Additionally, the same lectures were taught remotely to selected students of Ukrainian Physics and Mathematics Lyceum. 

The material has been designed to be accessible to people with solid background in high-school algebra and physics (mostly mechanics). Several years of teaching to a relatively diverse set of students proved that nearly all material can be efficiently absorbed by most, provided diligent work is done on exercises and problem. The last fact confirms a well-known truism: \emph{No real learning occurs without practice.}

Exercises are essential part of this book.  They are carefully selected to help readers get better understanding of the material and they are also fully solved. The difficulty of the exercises varies from simple to quite challenging.

This book \emph{is not a standard textbook}. It differs from many excellent introductions into Quantum Physics in that it lacks the breadth and rigor of the latter. However, this book serves a special purpose: It tries to act as the \emph{bridge} between  elementary and popular books and the more challenging college-level textbooks. 

If a picture is worth a thousand words, then a formula is worth a couple of hundred words. This book contains pictures and formulas aplenty. Hopefully, the readers for whom this book is intended will enjoy both.

Some sections are marked with an asterisk, for example
{\bf Transposition*}. Those sections contain material that is either
optional or a bit more advanced that usual. These sections can be
skipped without significant impact on the main message of the book.


\begin{center}
	{\bf At Any Cost}
\end{center}

The subtitle of this book has been inspired by the letter from Max Karl Ernst Ludwig Planck to an American physicist Robert Williams Wood. Describing his desperate attempts to explain the experimental results on the electromagnetic radiation from hot materials, Max Planck wrote\footnote{Source!} (italics are mine):
\begin{mybio}{Max Planck to Robert Wood}
	A theoretical interpretation therefore had to be found {\it at any cost}, no matter how high. It was clear to me that classical physics could offer no solution to this problem, and would have meant that all energy would eventually transfer from matter to radiation. ...This approach was opened to me by maintaining the two laws of thermodynamics. The two laws, it seems to me, must be upheld under all circumstances. For the rest, I was ready to sacrifice every one of my previous convictions about physical laws. ...[One] finds that the continuous loss of energy into radiation can be prevented by assuming that energy is forced at the outset to remain together in certain quanta. This was purely a formal assumption and I really did not give it much thought except that {\it no matter what the cost, I must bring about a positive result.}
	
\end{mybio}
Trying to provide a theoretical explanation at any cost, Max Planck introduced the idea of energy quanta, initiating the development of quantum ideas and becoming "the father of quantum physics."
