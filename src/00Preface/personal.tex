\graphicspath{{../00Preface/pics/}}

\section*{Personal Message}

\begin{wrapfigure}{l}{0.4\textwidth}
	\centering
	\includegraphics[width=1.0\linewidth]{author}
\end{wrapfigure}

\emph{Dear Reader}, thank you for picking up this book. Although we are not acquainted, I know a couple things about \emph{You}. First, you have a beautiful name, and second, you are curious about science and quantum physics. This is awesome!

Today is a great time for learning quantum physics. On the one hand, the abundance of educational resources -- both in printed and video format --  is impressive and, at times, confusing. On the other hand, time is a valuable resource and must be used wisely. I kept these thoughts in mind, when writing this book.

Respecting your time and effort, I am asking you the following: If at any point you find the book too challenging, unintersting, or not helpful -- \emph{please stop and put it away!} But \emph{do not give up} on learning quantum physics! Try other books.

There are so many great books and articles on quantum physics that simply listing them could fill the rest of the pages. Here is a short list of books on quantum physics that, in my opinion, would be great substitutes for mine:

\begin{enumerate}
	\item {\it Quantum Mechanics For Beginners} by Suhail Zubairy.
	\item {\it Quantum Mechanics: An Experimentalist's Approach} by Eugene Commins. 
\end{enumerate}


\subsubsection*{Personal Quirks}
In this book you will encounter several examples of notation that might appear strange. This is done on purpose.
One example is when instead of the monstrous sigma notation for the summation:
\[
S = \sum\limits_{i=0}^{i=N} a_i
\]
we will use 
\[
S = \int_i a_i\,\quad i=\overline{1, N}.
\]

\newpage