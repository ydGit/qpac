\graphicspath{{../07Implications/pics/}}

\chapter{Implications}\label{ch:Implications}

\lettrine[lines=2]{\color{darkocre}W}{e} are now ready to appreciate
the implications of quantum physics.

\begin{myprereq}{Prerequisite Knowledge}
	To fully understand the material of this chapter, readers should be comfortable with the following concepts:
	
	\begin{itemize}
		\item \phantom{phantom}
		\vspace{-0.5cm}
		\item State
		\item Dynamical equations
	\end{itemize}	
\end{myprereq}


\subsubsection*{$\delta$-Notation}\index{Notation!delta}
When a quantity $x$ changes by a tiny amount, we will denote the
change using small Greek letter $\delta$ (delta) as follows:
\[
\colorboxed{blue}{\delta x\textrm{ - tiny change of } x.}
\]
A convenient way to write all components of a second rank tensor is to
use table-like structure called \emph{matrix}.

\section*{Chapter Highlights}
{\setstretch{1.5}\chhc
	\it	
	\begin{itemize}
		\item Tensors find application in various areas of science and math.
		\item Geometrical properties of surfaces and spaces can be described
		using metric tensor.
		\item Physical properties of solids are often anisotropic -- depend on
		the direction of applied ``force''. Such properties are best
		described by various tensors: stress tensor, mobility tensor,
		piezoelectric tensor, and others.
		\item At the fundamental level electric and magnetic fields are united
		in a single physical object -- electromagnetic field. Electromagnetic
		field is described by an antisymmetric tensor of the second rank.
	\end{itemize}
	
}