\graphicspath{{../08Appendix/pics/}}


\chapter{Appendix}\label{ch:Appendix}

\lettrine[lines=2]{\color{darkocre}W}{e} are now ready to appreciate
the implications of quantum physics.

\section{Physics}
When a quantity $x$ changes by a tiny amount, we will denote the
change using small Greek letter $\delta$ (delta) as follows:
\[
\colorboxed{blue}{\delta x\textrm{ - tiny change of } x.}
\]
A convenient way to write all components of a second rank tensor is to
use table-like structure called \emph{matrix}.
\subsection{Black Body Radiation}

\subsection{Notation}
$K$ and $E_k$ -- Kinetic energy of a system.\\
$\Pi$ and $E_p$ -- Potential energy of a system.\\
$E$ -- Total mechanical energy ($E=E_K+E_P$) written in terms of velocity $v$ and position $x$.\\
$H$ -- Hamiltonian of a system: $H=K+\Pi$. Differs from $E$ because
kinetic energy written in terms of \emph{momentum} $p$ instead of velocity.\\
$L$ -- Lagrangian (Lagrange function) of a system: $L=E_K-E_p$. It is the ``imbalance''
of energies.\\
$\Delta x$ -- Change of a value of a variable $x$.\\
$\delta x$ -- ``Tiny'' change of a value of a variable $x$.\\
$\partial$ -- Rate of change.\\
$\partial_{t}$ -- Rate of change with respect to time.\\
$\partial_{x}$ -- Rate of change with respect to variable $x$
(e.g. position).\\
$\partial_{t}f$ -- Rate of change of $f$ with respect to $t$.\\
It means exactly the following
\[
\partial_t f = \frac{\delta f}{\delta t}=\frac{f(t+\delta t)-f(t)}{\delta t}\,.
\]\\
$\vec{\xi}$ -- State of a system in Hamiltonian dynamics. It is a vector
with components $\vec{\xi}=(x,p)$.\\
$\hat{J}$-- Operation (operator) of rotation by 90 degrees.\\
$\hat{R}(\theta)$ -- Operation (operator) of rotation by $\theta$.\\
$h$ -- Quantum of action (Planck's constant). In SI units its numerical
value is $h=6.626\times10^{-34}(J\cdot s)$.\\
$\hbar$ -- ``Reduced Planck's constant''. A convenience notation
for often used combination $\hbar=h/(2\pi)$.\\
$A$ -- Action.\\
$\Psi$ -- Quantum state.\\
$|\Psi\rangle$ -- Quantum state vector.\\
$\phi,\,\theta$ -- Angle variables.\\
$\omega$ -- Angular speed (also angular velocity). Often it has
the following meaning: $\omega=\partial_{t}\theta$ .\\
$\vec{e_{1}},\vec{e_{2}}$ -- Basis vectors. Usually they have unit
length and point in mutually perpendicular directions.\\
$z$ -- Arbitrary \emph{numeric} variable, $\vec{z}$ -- arbitrary
\emph{vector} variable, $\hat{z}$ -- arbitrary \emph{operator}.\\
$\overset{\circ}{A}$ -- Angstrom, a unit of length in the world
of atoms. $1\overset{\circ}{A}=10^{-9}(m)$. Hydrogen atom is about
$1\overset{\circ}{A}$ in diameter.\\
$c$ -- Speed of light in vacuum.\\
$\nu$ -- Frequency of oscillations measured as the number of
oscillations per second, in Hz.


\subsection{Physical Constants}
Below is the list of various physical constants used in these notes.\\
$q_e = 1.6\times 10^{-19}\,(C)$ -- Charge quantum (charge of an
electron).\\
$m_e = 9.1\times 10^{-31}\,(kg)$ -- rest-energy (aka mass) of an electron.\\
$k=\frac{1}{4\pi\epsilon_0} = 9\times 10^9\,(N\cdot m^2/C^2)$ --
Coulomb constant -- force between two unit charges 1 meter apart.\\
$10^{-9}$ s = 1 nanosecond -- the unit of time in atomic world. It is a ``heartbeat
of atoms''.\\
$1\, (eV) = q_e\, (J)$ -- 1 electron-volt. It is the kinetic energy
an electron would acquire when accelerated by a simply 1V battery. A
tiny value.\\
$m_ec^2/q_e= 0.5\, MeV$ -- rest-energy of an electron measured in
electron-volts. Roughly speaking, we will need half a million
1-volt batteries to accelerate an electron to make its kinetic energy
comparable to its rest-energy. \\
$k=100\,(N/m)$ is a spring constant of a spring that stretches by 0.1
of a meter when 1 kilogram mass is attached to it.\\



\section{Mathematics}
When a quantity $x$ changes by a tiny amount, we will denote the
change using small Greek letter $\delta$ (delta) as follows:
\[
\colorboxed{green}{\delta x\textrm{ - tiny change of } x.}
\]
A convenient way to write all components of a second rank tensor is to
use table-like structure called \emph{matrix}.

\subsection{Greek Alphabet}
\begin{table}[h]
  \begin{tabular}{l l c l l}
    \toprule
    $\mathrm{A}\, \alpha$ & alpha & & $\mathrm{B}\, \beta$ & beta\\
    $\Gamma\, \gamma$ & gamma & & $\Delta\, \delta$ & delta\\
    $\mathrm{E}\, \epsilon$ & epsilon & & $\mathrm{Z}\, \zeta$ & zeta\\
    $\mathrm{H}\, \eta$ & eta & & $\Theta\, \theta$ & theta\\
    $\mathrm{I}\, \iota$ & iota & & $\mathrm{K}\, \kappa$ & kappa\\
    $\Lambda\, \lambda$ & lambda & & $\mathrm{M}\, \mu$ & mu\\
    $\mathrm{N}\, \nu$ & nu & & $\Xi\, \xi$ & xi\\
    $\mathrm{O}\, \mathrm{o}$ & omicron & & $\Pi\, \pi$ & pi\\
    $\mathrm{P}\, \rho$ & rho & & $\Sigma\, \sigma$ & sigma\\
    $\mathrm{T}\, \tau$ & tau & & $\Upsilon\, \upsilon$ & upsilon\\
    $\Phi\, \phi$ & phi & & $\mathrm{X}\, \chi$ & chi\\
    $\Psi\, \psi$ & psi & & $\Omega\, \omega$ & omega\\
    \bottomrule
  \end{tabular}
  \caption{Greek Alphabet}
\end{table}

In mathematics most often we use $\theta$ and $\phi$ for
angles. Sometimes $\alpha$ and $\beta$ are also used. Occasionally
$\psi$ is used to denote angle.

In physics $\lambda$ is used to denote the wavelength of light, $\nu$
-- frequency in Hertz (periods of oscillations per second), $\omega$
-- angular speed (number of radians of rotation per second).

The symbols $\Psi$ and $\Phi$ are usually used to denote quantum state vectors.


\subsection{Available Environments}
Coloredboxed environment, with \textbackslash coloredboxed\{color\}\{ text \}:
\[
\colorboxed{red}{E=mc^2}.
\]

Bold text command inside math mode is \textbackslash btc\{ txt\}:

\[
\btc{max}\,x\,y
\]

Bold text with emphasis (italic) is done with \textbackslash bem\{text\}:
\bem{example of a very important piece of text}.



Grey bullet: \textbackslash tus:
\tus\tus\tus\tus

Quantum state is \textbackslash qs: $\qs{\psi}$ or better use ket and bra commands: $\ket{\phi}$ and $\bra{\psi}$.

Bracket and ketbra combinations using a single command: $\braket{\psi}{\phi}$ and $\ketbra{\psi}{\phi}$.

Operators are set with \textbackslash op\{name\}:
$\op{\rho}=\ketbra{\psi}{\psi}$.

\[
\oop{\rho} \ne \op{\rho}
\]

\subsubsection{Environments}

Prerequisite environment \textbackslash myprereq\{text\}:
\begin{myprereq}{Prerequisites}
	One, two, and three.
\end{myprereq}


Example environment \textbackslash myExample\{text\}:
\begin{myExample}
	This is a perfect example.
\end{myExample}

Analogy environment \textbackslash analogy\{text\}:
\begin{analogy}
	Consider the following analogy: A and B.
\end{analogy}

Definition environment \textbackslash mydef\{text\}:
\begin{mydef}{Kinetic Energy}
	Kinetic energy is the energy due to motion.
\end{mydef}

Reminder environment \textbackslash myrem\{text\}:
\begin{myrem}{Kinetic Energy}
	Recall that $E_k=mv^2/2$.
\end{myrem}

Bio environment \textbackslash mybio\{text\}:
\begin{mybio}{Max Planck}
	The full name of Max Planck is Max Karl Ernst Ludwig Planck.
\end{mybio}

\section{Temporary Stuff}
Work in progress sections will be kep here.

\subsection{Facts About Light}
Our understanding of the physical nature of light is the result of several centuries of studies. Visible, infra-red or ultra-violet  light, and microwave or gamma radiation are all various manifestations of \emph{electro-magnetic field} (EMF). It is one of the most technologically important and well-understood fields. 

\subsubsection*{Omnipresence}
The first important fact about EMF is that it is truly ubiqutous: There is no place in the universe free from EMF. Yes, there are regions of space (and time) where EMF is not excited, but EMF is still present there in its "unexcited" form, called \emph{EMF-vacuum}. This vacuum state is an important quantum state of EMF. It is discussed in more details in Section XXX.

\subsubsection*{Localizability}
The second important fact about EMF is its ability to be localized (concentrated) in finite regions of space, simetimes in small volumes. Put differently, EMF can be excited ("emitted") by relatively small "objects" (sources of EMF excitations) or absorbed by similar objects. Emission and absorption of light by atoms is a perfect example of this principle.

\begin{myrem}{Localized Excitations}
	We must clarify: It is the \emph{excitation of EMF} which is localized, \emph{not} EMF. The latter is everywhere, always.
\end{myrem}


\subsubsection*{Energy-Momentum}
Excitations of EMF propagate through space and possess the basic mechanical aspect -- \emph{energy-momentum}. When an atom emits a pulse of light it loses certain energy when and it also experiences a recoil, like a gun that fires a bullet. An EMF excitation carrying energy $E$ also carries momentum $p$. The two are related as follows:
\[
E = pc\,,
\]
where $c$ is the speed of light -- the speed of propagation of EMF through "empy" space. This formula is a special case of a more general relation between energy and momentum, derived in the special theory of relativity:
\[
E^2 = p^2 c^2 + m^2c^4\,.
\]
This is the relation between \emph{total} (kinetic and non-kinetic) energy of an object with mass $m$ and momentum $p$. It shows that even for an object at rest ($p=0$) it still possesses non-kinetic energy -- called rest-energy -- equal $E_0=mc^2$. For EMF excitations the total energy $E$ is \emph{purely due to motion} or, equivalently, EMF excitations are massless $m=0$.
\begin{myrem}{Photons}
	In common language, propagating EMF excitations are called \emph{photons}. It is sometimes said that photons with energy $E$ carry momentum $p=E/c$ and that photons are massless particles. It must be emphasized again that such language is not always helpful.
\end{myrem}

\subsubsection*{Colors}
Beyond energy-momentum, EMF excitations exhibit various "colors" -- both visible and invisible to the human eye.

\subsubsection*{Timeless Law}
The laws that describe the behavior of electromagnetic radition in space and time are expressed by the equations which show no preference to the direction of time flow.

