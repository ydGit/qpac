\chapter{Applications}\label{ch:Applications}
\lettrine[lines=2]{\color{darkocre}W}{e} are now ready to appreciate
how tensors are used in ``real life''. In this chapter we will
encounter examples of
tensors that are used in mathematics, physics, and engineering. Before we get
to the examples of tensors, one more helpful notation must be explained.

\subsubsection*{$\delta$-Notation}\index{Notation!delta}
When a quantity $x$ changes by a tiny amount, we will denote the
change using small Greek letter $\delta$ (delta) as follows:
\[
\colorboxed{red}{\delta x\textrm{ - tiny change of } x.}
\]
For example, for the earth going around the sun in 365 days, one
second elapsed on a clock can be considered a tiny change $\delta
t$. When a drop of water falls into a nearly full bucket the mass
of the latter changes by a tiny amount $\delta m$, and so on.

That's all there is to
$\delta$-notation. We are not going into the realm of calculus, where
mathematicians talk about infinitesimal quantities and limits; we
will be simply using ``tiny changes.'' Now on to tensors.

\section{Famous Tensors}
We will study several examples of tensors that readers most likely
encounter in geometry, physics, and engineering. The material in the
previous chapters should be enough to prepare a reader to deal with
tensors of any kind. However, we will limit considerations to simple
tensors of lower ranks.

\subsection{Metric Tensor}\index{Tensor!metric}
\emph{Metric tensor} is used to determine distances between pairs of
points in space. A distance between two points is equal to the length of a
vector connecting them, as shown in the Figure \ref{fig:distanceVector}. For a vector
\[
\vec{d} = \vec{b} - \vec{a}\,,\quad d^i = b^i - a^i
\]
its length squared is given by the scalar product
\[
d^2 = \vec{d}\cdot\vec{d} = \op{\sigma}\, \vec{d}\,\vec{d}\,.
\]


Using components in arbitrary basis (not orthonormal), the length squared is written as
\[
d^2 = \op{\sigma}\, (d^i\vec{e}_i)\,(d^j\vec{e}_j) = d^id^j\,(\op{\sigma}\,\vec{e}_i\,\vec{e}_j)\,.
\]

The set of values
\[
\eta_{ij} = \op{\sigma}\,\vec{e}_i\,\vec{e}_j
\]
corresponds to the components of a special tensor -- \emph{metric
tensor}. The transformation rule of these components is easily found
by expanding ``old'' basis vectors in terms of the ``new'' (primed)
basis, and using the linearity of $\op{\sigma}$:
\[
\eta'_{ij} = \op{\sigma}\,\vecp{e}_i\,\vecp{e}_j =
\op{\sigma}\,(E_{im}\vec{e}_m)\,(E_{jn}\vec{e}_n) = E_{im}E_{jn}\eta_{mn}\,.
\]
This is the transformation rule of a covariant-covariant tensor of the
second rank. The metric tensor has to be of this kind since it maps a
contravariant-contravariant tensor
\[
\vec{d}\otimes\vec{d}\,,\quad d^id^j
\]
into a scalar. Each index of the metric tensor must transform in a way
that ``compensates'' the contravariant transformation of $\vec{d}$ in the
tensor product $\vec{d}\otimes\vec{d}$. This is analogous to how a
covariant vector $\vecl{b}$ maps a contravariant vector $\vec{a}$ into
a number:
\[
\vecl{b}\vec{a}\,\longrightarrow\, x\,
\]
\[
\eta\,(\vec{a}\otimes\vec{a})\,\longrightarrow\, y\,.
\]
%\begin{tcolorbox}[colback=white!85!ocre, title=Problem]
\begin{exercise}\label{exe:metricTensorScalarProduct}
Although the primary use of the metric tensor is to calculate
distances between a pair of points connected by a vector $\vec{d}$,
it can be applied to any pair of contravariant vectors:
\[
\eta\,(\vec{a}\otimes\vec{b})\,.
\]
a) What is the meaning of this operation? b) What are the components of the
metric tensor in orthonormal basis?
\end{exercise}
%\end{tcolorbox}

In most elementary problems of geometry and physics, a coordinate
system in a plane is Cartesian and basis vectors are
orthonormal\index{Basis!orthonormal}. As
the result, the components of the metric tensor are trivial -- zeros
and ones -- and are the same everywhere in the plane.

When non-Cartesian coordinates are used in a plane, e.g. polar
coordinates\index{Coordinate!system!polar} shown in the Figure
\ref{fig:polarCartesianMetric}, the
basis vectors are aligned with
the coordinate grid\footnote{In principle, it is possible to have
basis vectors ``decoupled'' from the coordinate system, but this is
not very convenient.} and have different orientation in different
points. In this case the components of the metric tensor $\eta_{ij}$
 change from point to point to ensure that the lengths of the vector
 $\vec{d}$
 \[
 d^2 = \eta_{ij}d^id^j
 \]
 remains the same.

Moreover, for surfaces more sophisticated than a plane
(e.g., sphere, paraboloid, saddle-like surface, and myriad of others),
it is impossible to use coordinate system and basis such that the metric
tensor is constant. The components of the metric tensor will vary
across the surface to reflect real, and not a merely ``coordinate
induced'', difference of a surface from a plane. In other words,
variation of the components of the metric tensor indicate that the
surface is \emph{curved}.


As an example, consider two dimensional surface of a sphere,
shown in the Figure \ref{fig:sphericalMetric}. Each point on the
surface can be located using a pair of coordinates -- the angle
$\theta=x^1$ complimentary to the latitude, and the longitude angle
$\phi=x^2$. If a pair of close points on the sphere have coordinates
\begin{eqnarray*}
1 & \quad\longrightarrow \quad & (\theta, \phi),\\
2 & \quad\longrightarrow \quad & (\theta+\delta\theta, \phi+\delta\phi),
\end{eqnarray*}
then the distance squared between these points is given by
\[
d^2 = R^2(\delta\theta)^2 + R^2\sin^2\theta(\delta \phi)^2.
\]
This formula is obtained by applying Pythagoras theorem to the
tiny right triangle with the sides indicated using arrows in the Figure
\ref{fig:sphericalMetric}. The length of the side resulting from the
change of the coordinate $\phi$ is $r\delta\phi =
R\sin\theta\delta\phi$; the length of the side resulting from the
change of the coordinate $\theta$ equals $R\delta\theta$.

Using the uniform notation for coordinates, the distance squared is
written as
\[
d^2 = R^2(\delta x^1)^2 + R^2\sin^2 x^1(\delta x^2)^2\,.
\]
Comparing this to the Cartesian expression
$d^2 = (\delta x^1)^2 +(\delta x^2)^2$, we can see that not all
components of the metric tensor in the
spherical coordinate basis are constant. Namely, the component
$\eta_{22}=R^2\sin^2 x^1$ depends on the coordinate $x^1=\theta$.

\subsection{Anisotropy Tensor}
\emph{Anisotropy tensor} is a general term for various tensors used in
physics to describe properties of materials like crystalline
solids. Many physical properties -- mechanical, optical, electronic, thermal --
describe the response of material to the external
``forces'' or perturbations. Mathematical description of such
responses requires tensors.

To understand the general idea, let us consider a simple situation,
depicted in the Figure \ref{fig:anisotropicTensor}. Suppose that a
tree bends in the wind, so that when the wind blows in the direction
of the $x$ axis, the displacement of the tree-top is also along the
$x$ axis, with the magnitude proportional to the magnitude of the
wind's velocity $\vec{v}$:
\[
\vec{d} = d_x\vec{u}_1 = Av_x \vec{u}_1 = A\vec{v}\,.
\]
Next, suppose that when the wind blows along the $y$ axis, the
tree-top is also displaced in the direction of the $y$-axis:
\[
\vec{d} = d_y\vec{u}_2 = Bv_y \vec{u}_2 = B\vec{v}\,.
\]
In both cases the displacement vector $\vec{d}$ is parallel to the vector of
wind's velocity.

For a general direction of the wind, the magnitude of the tree-top
displacement will be proportional to the magnitude of the wind's
velocity, but \emph{the direction of the displacement will differ from
the direction of the wind}:
\[
d\propto v\,,\quad \vec{d}\nparallel\vec{v}\,.
\]

Indeed, for a wind vector
\[
\vec{v} = v_x\vec{u}_1 + v_y\vec{u}_2 = \vec{v}_1 + \vec{v}_2\,,
\]
the ``response'' of the tree-top will be different for different
components of the wind vector:
\[
\vec{d} = \vec{d}_1 + \vec{d}_2 = A\vec{v}_1 + B\vec{v}_2 = Av\cos\theta\vec{u}_1 + Bv\sin\theta\vec{u}_2=d_x\vec{u}_1+d_x\vec{u}_2\,.
\]
Clearly,
\[
\frac{d_y}{d_x} = \frac{B\sin\theta}{A\cos\theta} \ne \tan\theta\,.
\]
Thus, although the displacement magnitude is still proportional to the
magnitude of the wind's velocity, the direction of the displacement no
longer coincides with the direction of the wind. This fact can be
expressed using tensor notation:
\begin{equation}
  d^i = T^i_{\phantom{x}j}v^j\,.
  \label{eq:displacementTensor}
\end{equation}
In the special coordinates, considered at the beginning of this
problem, the components of the tensor $T$ are simple:
\[
T^1_{\phantom{x}1} = A\,\,, T^1_{\phantom{x}2} = 0\,\,,
T^2_{\phantom{x}1} = 0\,\,, T^2_{\phantom{x}2} = B\,\,.
\]
In all other coordinate systems and bases, the components of the
``response tensor'' $T$ can be found using the transformation rule for
the contravariant-covariant tensor of the second rank.


Expressions similar to the equation (\ref{eq:displacementTensor}) can
be written for a variety of physical phenomena. We will consider
a couple of examples next.

\begin{flushleft}
{\bf Mechanics: Stress Tensor}\index{Tensor!stress}
\end{flushleft}
Mechanics of elastic media uses many tensor tools. One of the basic
tensors is the \emph{stress tensor.}\footnote{Also known as
\emph{Cauchy stress tensor} or \emph{true stress tensor.}} This tensor
describes the distribution of mechanical stress inside a deformed
elastic body, as illustrated in the Figure \ref{fig:stressTensor}.


An elastic ball, shown in the Figure \ref{fig:stressTensor}(a), can be
squeezed by external forces, resulting in the change of shape
(\emph{deformation}) and the appearance of a mechanical stress inside
the ball; see Figure \ref{fig:stressTensor}(b). In general, the induced
stress will change from point to point inside the deformed ball. To
describe the stress in a given point $P$, we can imagine that a small
part of the body is removed, leaving a tiny square-shaped hole. If
nothing is done, the empty part of the ball around the point $P$ will
not remain square, due to the ``forces'' acting within the body and at
the boundary of the hole. To keep the hole square, we must compensate
the forces due to mechanical stress and apply the balancing forces
$\vec{f}_1,\,\vec{f}_2,\,\vec{f}_3,\,\vec{f}_4$ to each side of the
square\footnote{For a three dimensional ball the shape of the hole
will be a cube, and the number of forces will be 8 -- one for each
face of the cube.}. Only two such forces are shown in the Figure
\ref{fig:stressTensor}(b) for simplicity.

The direction and magnitude of a force needed for a given side can be
found as follows: First, find the unit-length vector $\vec{e}_i$
perpendicular to the side. Second, calculate the force using the
Cauchy stress tensor:
\[
\vec{f}_i = \op{\sigma}\,\vec{e}_i\,.
\]
The traditional notation for mechanical stress tensors is Greek letter
sigma --$\sigma$. It should not be confused with out notation for
dol-operator defined earlier (see Section \ref{sec:Dol}).

It is easy to understand why the ``balancing forces'' are not, in
general, pushing perpendicular to the sides of the square. The idea is
illustrated in the Figure \ref{fig:stressTensor}(c). An elastic square
(cube) can be deformed in two basic ways: 1) A square can be squeezed
by forces perpendicular to the sides (\emph{normal stress}); 2) A
square can be deformed into a parallelogram by forces parallel to the
sides (\emph{shear stress}). Both types of stress can exist at the
same time, resulting from forces directed at an arbitrary angle
relative to the vector $\vec{e}_i$ perpendicular to the sides.

\begin{flushleft}
{\bf Electronics: Mobility Tensor}\index{Tensor!mobility}

Many materials conduct electric current. An important
characteristic of such a material is its \emph{resistance}. When a
voltage $V$ is applied to a piece of conducting material, the current
$I$ will flow between the terminals, as shown in the Figure
\ref{fig:mobilityTensor}.
\end{flushleft}


The basic law that relates the voltage $V$ and the current $I$ between the
terminals is \emph{Ohm's law}\index{Ohm's law}:
\[
V = I R\,.
\]
Here $R$ is the electric resistance of a given piece of material.

For the same material, currents flowing in different directions may
experience different resistances even for the same geometrical
shape. In the example shown in the Figure
\ref{fig:mobilityTensor}(a,b), for currents flowing horizontally and
vertically through a square we can write
\[
V = I_1 R_1\,\textrm{ and }\,V = I_2 R_2\,.
\]

We can take another view on the movement of electric charge through the
material if we rewrite Ohm's law as follows:
\[
I = GV\,.
\]
Here instead of resistance we use an equally useful physical parameter
called \emph{conductance} $G$. In certain sense, conductance is more
fundamental since it is closely related to basic physical laws that
govern the motion of electric charges.

Electric current is the flow of a large number of charge carriers, such
as electrons or ions, shown as red dots in the Figure
\ref{fig:mobilityTensor}(a,b). The current $I$ is proportional to the
average speed $u$ of the carriers through the material:
\[
I \propto u\,.
\]
The carriers, in their turn, move because there is an electric field
$E$ between the terminals due to applied voltage $V$. The average
speed $u$ of charge carriers is often simply
proportional to the electric field:
\[
u = \mu E\,,
\]
where the coefficient $\mu$ is called \emph{mobility} of charge carriers.

Now for anisotropic materials, the relation between the average
velocity $\vec{u}$ of charge carriers and the applied electric field
$\vec{E}$ can be written using the concept of \emph{mobility tensor}:
\[
\vec{u} = \op{\mu}\, \vec{E}\,.
\]
The mobility tensor expresses how easy it is to make electrons move in
a given direction by applying an external electric field $\vec{E}$.

Let us summarize: Applying voltage between terminals creates an
electric field $\vec{E}$ in a given direction. The electric field is
proportional to the voltage between the terminals: $E=V/d$. The
electric field leads to the ``mass migration'' of charge carriers with
the average speed
\[
u=\mu E = \mu V/d\,.
\]
This type of motion is called electric current:
\[
I\propto u\quad\longrightarrow\quad I\propto \frac{\mu}{d}V\,.
\]
From the last expression we can see how the relationship $I=GV$ or Ohm's law $V=IR$ appear.
Furthermore, because the mobility $\op{\mu}$ is in general a tensor, the
measured resistance of a given piece of material may be different for
different direction of applied voltage drop $V$.

\begin{mybio}{Anisotropy Tensors in Physics}\index{Tensor!anisotropy}
Besides two examples of tensors (stress and mobility) given above,
there are many other tensors used in physics. Some tensors are similar
to stress and mobility tensors in the sense that they express linear
relationship between ``action'' ($\vec{a}$) and ``response''
($\vec{r}$) vectors
\[
\vec{r} = \op{t}\,\vec{a}\,.
\]
But more advanced tensors are also used to express linear
relationships between more simple tensors and vectors. For example, in
certain materials mechanical stress can lead to separation of electric
charges and thus create voltage drop between different points of the
body. This phenomenon is known as \emph{piezoelectric effect}. Now if
we characterize induced charge separation using a vector $\vec{p}=p^i$,
then we can write using index notation
\[
p^i = d^{ijk}\sigma_{jk}\,,
\]
where $\sigma_{jk}$ are the components of the stress tensor described
above, and $d^{ijk}$ -- piezoelectric tensor of the third rank (three
indices!)

For the reader interested in more examples and details, the book
{\it ``Physical Properties of Crystals: Their Representation by Tensors and
Matrices''} by  J. F. Ney is highly recommended.
\end{mybio}
As the last example of tensors in physics, we will consider a more
fundamental case from field theory.

\subsection{Electromagnetic Tensor}\index{Tensor!electromagnetic}
In applied physics and engineering one works with electric and
magnetic fields that are described using two \emph{different} physical vector
quantities: $\vec{E}$ -- for electric field strength, and $\vec{B}$
-- for magnetic field strength.

When a charged particle, say an electron, is placed in electric field,
the latter acts on that particle with force proportional to the field
strength:
\[
F_e = qE\,,
\]
where $q$ is the charge of the particle, $F_e$ denotes the force
due to the electric field $\vec{E}$.

When the same charged particle is \emph{moving} in a magnetic field, the
latter acts on the particle with the force proportional to the
magnetic field strength:
\[
F_m = qvB\,,
\]
where $v$ is the speed of the charge particle, $F_m$ denotes the force
due to the magnetic field $\vec{B}$. The difference between the
effects of electric and magnetic fields on a charged particle is
illustrated in the Figure \ref{fig:fieldsElectricMagnetic}.

The distinction between electric and magnetic fields is technical and
\emph{not fundamental}. Figuratively speaking, electric field differs
from magnetic field to the same degree as rest differs from uniform
motion. Electric and magnetic fields are different aspects of the same
physical entity -- \emph{electromagnetic field}.

From the expressions for the electric and magnetic forces $F_e$ and
$F_m$ we can see that physical quantities $E$ and $B$ have different
units of measurement -- the fact which upsets some physicists. They
note: If electric and magnetic fiels are different aspects of the
\emph{same physics object}, they must be measured using the same
units, similar to how we measure height and width of a building using
the same units of length.

The way to fix the issue with different units for electric and
magnetic fields, is to change the way we measure...\emph{velocity}! Nature
provides us with a special standard of speed -- the speed of light in
vacuum, denoted  as $c$. The speed of light in vacuum is a
``nature-made'' absolute quantity, in contrast to such human-made standards as units of
length (meter) or time (second). This is why in fundamental physical
theories,
including the theory of electromagnetic field, it is wise to specify
all speeds as fractions of the speed of light.

Thus, in physical formulas, instead of writing $v$ as meters per
second, we should use a ``normalized'' quantity:
\[
\bar{v} = v / c\quad\rightarrow\quad v = \bar{v}c\,.
\]
Once we apply this approach to the expression of the magnetic force,
we obtain
\[
F_m = qvB = q\bar{v}(cB)\,.
\]
Now we can see that the quantities $E$ and $cB$ have the same
physical meaning -- \emph{the force per unit charge}. It is these
physical quantities that should be used to describe different aspects
fo the same electromagnetic field. We will denote them as follows:
\[
\vec{E} = \vec{\mathcal{E}}=\mathcal{E}^i\,,\quad c\vec{B} =
\vec{\mathcal{B}}=\mathcal{B}^j\,.
\]

\begin{mybio}{Electromagnetic Tensor}
A deep and beautiful discovery of the theory of electromagnetic field
can now be stated: Electric and magnetic fields
$\mathcal{E}^i$ and $\mathcal{B}^j$ are not separate \emph{vector}
quantities, they are, in fact, represent certain
\emph{components of a tensor} that describes electromagnetic field.

This tensor is conventionally written as $F^{\mu\nu}$ ($F$ here stands
for \emph{field}, not force!) $F^{\mu\nu}$ is a second rank tensor.
The indices $\mu$ and $\nu$ run from $0$ to $3$.

The relationships between the ``usual'' electric field and the
electromagnetic tensor are given by
\[
\mathcal{E}^i = F^{i0}\,,\quad i=1,2,3\,.
\]
The relationships between the ``usual'' magnetic field and the
electromagnetic tensor can be written in the following way:
\[
\mathcal{B}^1 = F^{32}\,,\mathcal{B}^2 = F^{13}\,,\mathcal{B}^3 = F^{21}\,.
\]
\end{mybio}

A convenient way to write all components of a second rank tensor is to
use table-like structure called \emph{matrix}\index{Matrix}:
\[
F^{\mu\nu}=
\begin{pmatrix}
  F^{00} & F^{01} & F^{02} & F^{03}\\
  F^{10} & F^{11} & F^{12} & F^{13}\\
  F^{20} & F^{21} & F^{22} & F^{23}\\
  F^{30} & F^{31} & F^{32} & F^{33}
\end{pmatrix}\,.
\]
In the matrix, the first index $\mu$ of $F^{\mu\nu}$ corresponds to
the row, while the second index $\nu$ corresponds to the column. Both
rows and columns are enumerated from $0$ to $3$.

Using matrix form, we can write the electromagnetic tensor in terms
of the electric and magnetic fields:
\[
F^{\mu\nu}=
\begin{pmatrix}
  0 & -\mathcal{E}^1 & -\mathcal{E}^2 & -\mathcal{E}^3\\
  \mathcal{E}^1 & 0 & -\mathcal{B}^3 & \mathcal{B}^2\\
  \mathcal{E}^2 & \mathcal{B}^3 & 0 & -\mathcal{B}^1\\
  \mathcal{E}^3 & -\mathcal{B}^2 & \mathcal{B}^1 & 0
\end{pmatrix}\,.
\]
The last expression makes apparent two features of electromagnetic
tensor components. First, all diagonal elements vanish:
\[
F^{00} = F^{11} = F^{22} = F^{33} = 0\,.
\]
Second,
\[
F^{\mu\nu} = -F^{\nu\mu}\,,
\]
a property known as \emph{antisymmetry}. This property requires all
diagonal elements to be equal zero.
\begin{mybio}{Electromagnetic Tensor Components}
  The first kind of tensor of the second rank that we encountered was
  a linear
  operator $\op{L}$. The components of any linear operator are given
  relative to some basis and the components specify how the operator
  transforms basis vectors:
  \[
  \op{L}\,\vec{e}_i = L_{ij}\vec{e}_j\,.
  \]
  What is the basis used to express components of electromagnetic
  tensor $F^{\mu\nu}$?

  Electromagnetic tensor is a physical operator which is used to express
  the action of electromagnetic field on a moving charged particle.
  Components of electromagnetic tensor connect special versions
  of velocity ($v_\nu$) and force ($f^\mu$) acting on a charged particle:
  \[
  f^\mu = qF^{\mu\nu}v_\nu\,.
  \]
  Without going into details, we will note that in the left-hand side
  of this equation we have a four-component force, and on the
  right-hand side we have both electric and magnetic effects combined
  in a single tensor.
\end{mybio}


\section*{Chapter Highlights}
{\setstretch{1.5}\chhc
  \it
  \small
\begin{itemize}
\item Tensors find application in various areas of science and math.
\item Geometrical properties of surfaces and spaces can be described
  using metric tensor.
\item Physical properties of solids are often anisotropic -- depend on
  the direction of applied ``force''. Such properties are best
  described by various tensors: stress tensor, mobility tensor,
  piezoelectric tensor, and others.
\item At the fundamental level electric and magnetic fields are united
  in a single physical object -- electromagnetic field. Electromagnetic
  field is described by an antisymmetric tensor of the second rank.
\item The concept of linear operators, and in particular of the
  rotation operator $\op{J}$, can be used to extend the numbers from a
  number line to the number plane and arrive at complex numbers (or
  compound numbers, as we called them).
\item Operators and compound numbers are used in many physical
  theories, and play an especially important role in Hamiltonian
  dynamics and quantum mechanics.
\end{itemize}

}
