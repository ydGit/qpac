\chapter{Applications}\label{ch:Applications}
\lettrine[lines=2]{\color{darkocre}W}{e} are now ready to appreciate
how tensors are used in ``real life''. In this chapter we will
encounter examples of
tensors that are used in mathematics, physics, and engineering. 

\begin{myprereq}{Prerequisite Knowledge}
	To fully understand the material of this chapter, readers should be comfortable with the following concepts:
	
	\begin{itemize}
		\item \phantom{phantom}
		\vspace{-0.5cm}
		\item State
		\item Dynamical equations
	\end{itemize}	
\end{myprereq}

\section{Hydrogen-like Atoms}
\[
\ketbra{\alpha}{\beta}
\]
\[
E_n = -\frac{E_i}{n^2}\,.
\]

\section{Quantum Dots}
\[
\ketbra{\alpha}{\beta}
\]
\[
E_n = -\frac{E_i}{n^2}\,.
\]

\section{Spontaneous Emission}
\[
\ketbra{\alpha}{\beta}
\]
\[
E_n = -\frac{E_i}{n^2}\,.
\]

\section{Stimulated Emission}
\[
\ketbra{\alpha}{\beta}
\]
\[
E_n = -\frac{E_i}{n^2}\,.
\]

\section{Lasers}
\[
\ketbra{\alpha}{\beta}
\]
\[
E_n = -\frac{E_i}{n^2}\,.
\]

\section{Photoeffect}
\[
\ketbra{\alpha}{\beta}
\]
\[
E_n = -\frac{E_i}{n^2}\,.
\]

\section{Black Body Radiation}
\[
\ketbra{\alpha}{\beta}
\]
\[
E_n = -\frac{E_i}{n^2}\,.
\]

\section{Conductors}
\[
\ketbra{\alpha}{\beta}
\]
\[
E_n = -\frac{E_i}{n^2}\,.
\]

\section{Entanglement}
\[
\ketbra{\alpha}{\beta}
\]
\[
E_n = -\frac{E_i}{n^2}\,.
\]

\subsubsection*{$\delta$-Notation}\index{Notation!delta}
When a quantity $x$ changes by a tiny amount, we will denote the
change using small Greek letter $\delta$ (delta) as follows:
\[
\colorboxed{red}{\delta x\textrm{ - tiny change of } x.}
\]
A convenient way to write all components of a second rank tensor is to
use table-like structure called \emph{matrix}\index{Matrix}:
\[
F^{\mu\nu}=
\begin{pmatrix}
  F^{00} & F^{01} & F^{02} & F^{03}\\
  F^{10} & F^{11} & F^{12} & F^{13}\\
  F^{20} & F^{21} & F^{22} & F^{23}\\
  F^{30} & F^{31} & F^{32} & F^{33}
\end{pmatrix}\,.
\]
In the matrix, the first index $\mu$ of $F^{\mu\nu}$ corresponds to
the row, while the second index $\nu$ corresponds to the column. Both
rows and columns are enumerated from $0$ to $3$.

Using matrix form, we can write the electromagnetic tensor in terms
of the electric and magnetic fields:
\[
F^{\mu\nu}=
\begin{pmatrix}
  0 & -\mathcal{E}^1 & -\mathcal{E}^2 & -\mathcal{E}^3\\
  \mathcal{E}^1 & 0 & -\mathcal{B}^3 & \mathcal{B}^2\\
  \mathcal{E}^2 & \mathcal{B}^3 & 0 & -\mathcal{B}^1\\
  \mathcal{E}^3 & -\mathcal{B}^2 & \mathcal{B}^1 & 0
\end{pmatrix}\,.
\]


\section*{Chapter Highlights}
{\setstretch{1.5}\chhc
  \it
\begin{itemize}
\item Tensors find application in various areas of science and math.
\item Geometrical properties of surfaces and spaces can be described
  using metric tensor.
\item Physical properties of solids are often anisotropic -- depend on
  the direction of applied ``force''. Such properties are best
  described by various tensors: stress tensor, mobility tensor,
  piezoelectric tensor, and others.
\item At the fundamental level electric and magnetic fields are united
  in a single physical object -- electromagnetic field. Electromagnetic
  field is described by an antisymmetric tensor of the second rank.
\end{itemize}

}
